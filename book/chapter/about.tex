\chapter{简介}
\section{关于这本书}
这本书
\section{关于作者}
\section{感谢}

\chapter{Why Gentoo}
\section{什么是Linux}
\section{关于GNU/Linux发行版}

\chapter{文件系统}
\section{VFS}
\section{ext2}
\section{ext4}
\section{xfs}
\section{zfs}
\section{btrfs}

\chapter{缓存}
\section{bcache}
\section{lvm-cache}
\section{dm-cache}
\chapter{GPG}
\section{什么是GPG}
\section{定义你的id}

\chapter{LUKS}
\section{VFS 和 LUKS}

\chapter{组合说明}
\section{Less sometimes is more}
\section{极致的安全}

\chapter{准备环境}
\section{环境说明}
\subsection{硬件环境}
这次安装的硬件环境 \ref{table:1} 。

\begin{table}[h!]
\centering
\begin{tabular}{||c| c| c| c|c||} 
 \hline
 CPU  & Memory  & Hard Disk & Wireless & Boot Firmware\\ [1ex]

 \hline
 i5-2450M & 2G*2    & 90G SSD*1 & AR9380 & coreboot\\
 \hline
\end{tabular}
\caption{ThinkPad X220 }
\label{table:1}
\end{table}

%\begin{tabular}{ |p|p|p|p|p|  }
% \hline
% \multicolumn{5}{|c|}{ThinkPad X220} \\
% \hline
% CPU  & Memory  & Hard Disk & Wireless & Boot Firmware\\
% \hline
% i5-2450M & 2G*2    & 90G SSD*1 & AR9380 & coreboot\\
% \hline
% \end{tabular}

同时需要以下设备
\begin{enumerate}
\item U盘*2(一个存储key)
\item 一台工作正常的笔记本
\item 可以访问Google
  \begin{itemize}
    \item 出现问题可以方便排查问题
  \end{itemize}
\item 稳定的网络
  \item \href{https://github.com/ryanhanwu/How-To-Ask-Questions-The-Smart-Way/blob/master/README-zh_CN.md}{提问的艺术}
\end{enumerate}



\subsection{准备知识}
\href{https://wiki.gentoo.org/wiki/Eselect}{Eselect}是一个很方便配置系统的工具。

\subsection{准备Gentoo Linux安装介质}
Gentoo有两种安装iso一种是最小化的iso叫做install-x86-minimal-<release>.iso另外一种是Gentoo livecd 有一个kde的桌面环境方便联网查询问题安装,我这里使用的是最小化的安装介质来安装。
\subsection{stage文件}
stage3压缩包是一个包含有最小化Gentoo环境的文件,可用来按照本手册介绍继续安装Gentoo。以前的Gentoo手册描述了使用三个 stage tarballs的其中一个来进行安装。现在Gentoo仍然提供stage1和stage2的压缩包,但是官方安装方法只使用stage3压缩包。也可以看我的另外cataly构建系统文章。


\subsection{下载安装iso与stage文件}
\subsubsection{下载iso}
Gentoo Linux使用最小化安装CD做为默认安装媒介,它带有一个非常小的可引导的Gentoo Linux环境。此环境包含所有正确的安装工具. CD镜像本身可以从
\href{http://distfiles.gentoo.org/releases/amd64/autobuilds/}{官方下载页面}下载(推荐) 或者是下面的镜像站\href{https://mirror.tuna.tsinghua.edu.cn/gentoo/releases/amd64/autobuilds/}{清华大学源}进行下载。

在这些镜像站上,最小化安装CD可以通过以下方式找到:
\begin{enumerate}
\item 进入 releases/ 目录
\item 选择对应的架构,我们的是amd64
\item 选择autobuilds目录
\item 对于amd64和x86的平台的用户需要选择 current-install-amd64-minimal/ 或 current-install-x86-minimal/ 目录。如果需要所有其它平台的,请进入 current-iso/ 目录。
\end{enumerate}

下载之后要对下载好的iso文件进行校验防止到时候出现无法使用的问题。


\subsubsection{下载stage3文件}
进入镜像站的/gentoo/releases/amd64/autobuilds/目录
如果你对systemd没有则进入current-stage3-amd64/目录选择最新的stage3下载到本地。


\subsection{在Linux下制作安装介质}

\begin{lstlisting}[language=Bash]
  dd if=gentoo.iso of=/dev/sdb bs=10M
\end{lstlisting}
\subsection{在Windows上准备安装介质}
可以使用软碟通之类的软件进行刻录
