% XeLaTeX Books
\documentclass{book}
% Packages list
% Chinese support
\usepackage{ctex}
% Code environment
\usepackage{listings}
\usepackage[utf8]{inputenc}
% \usepackage[chinese]{babel}
% Color
\usepackage{xcolor}
% Hyperref
\usepackage{hyperref}
% Own color
\definecolor{codegreen}{rgb}{0,0.6,0}
\definecolor{codegray}{rgb}{0.5,0.5,0.5}
\definecolor{codepurple}{rgb}{0.58,0,0.82}
\definecolor{backcolour}{rgb}{0.95,0.95,0.92}
% listings config
\lstdefinestyle{mystyle}{
    backgroundcolor=\color{backcolour},   
    commentstyle=\color{codegreen},
    keywordstyle=\color{magenta},
    numberstyle=\tiny\color{codegray},
    stringstyle=\color{codepurple},
    basicstyle=\ttfamily\footnotesize,
    breakatwhitespace=false,         
    breaklines=true,                 
    captionpos=b,                    
    keepspaces=true,                 
    numbers=left,                    
    numbersep=5pt,                  
    showspaces=false,                
    showstringspaces=false,
    showtabs=false,                  
    tabsize=2
}
 
\lstset{style=mystyle}

% Own Hypersetup
\hypersetup{
    colorlinks=true,
    linkcolor=blue,
    filecolor=magenta,      
    urlcolor=cyan,
}
\urlstyle{same}
% Author
\author{Lily Su}
\date{\today}
\begin{document}

\title{Gentoo 安装手册}
\maketitle
%
\frontmatter
\tableofcontents
%
\chapter{准备环境}
\section{环境说明}
\subsection{硬件环境}
\begin{enumerate}
\item U盘*1
\item 一台工作正常的笔记本
\item 可以访问Google
  \begin{itemize}
    \item 出现问题可以方便排查问题
  \end{itemize}
\item 稳定的网络
  \item \href{https://github.com/ryanhanwu/How-To-Ask-Questions-The-Smart-Way/blob/master/README-zh_CN.md}{提问的艺术}
\end{enumerate}



\subsection{准备知识}
\href{https://wiki.gentoo.org/wiki/Eselect}{Eselect}是一个很方便配置系统的工具。

\subsection{准备Gentoo Linux安装介质}
Gentoo有两种安装iso一种是最小化的iso叫做install-x86-minimal-<release>.iso另外一种是Gentoo livecd 有一个kde的桌面环境方便联网查询问题安装,我这里使用的是最小化的安装介质来安装。
\subsection{stage文件}
stage3压缩包是一个包含有最小化Gentoo环境的文件,可用来按照本手册介绍继续安装Gentoo。以前的Gentoo手册描述了使用三个 stage tarballs的其中一个来进行安装。现在Gentoo仍然提供stage1和stage2的压缩包,但是官方安装方法只使用stage3压缩包。也可以看我的另外cataly构建系统文章。


\subsection{下载安装iso与stage文件}
\subsubsection{下载iso}
Gentoo Linux使用最小化安装CD做为默认安装媒介,它带有一个非常小的可引导的Gentoo Linux环境。此环境包含所有正确的安装工具. CD镜像本身可以从
\href{http://distfiles.gentoo.org/releases/amd64/autobuilds/}{官方下载页面}下载(推荐) 或者是下面的镜像站\href{https://mirror.tuna.tsinghua.edu.cn/gentoo/releases/amd64/autobuilds/}{清华大学源}进行下载。

在这些镜像站上,最小化安装CD可以通过以下方式找到:
\begin{enumerate}
\item 进入 releases/ 目录
\item 选择对应的架构,我们的是amd64
\item 选择autobuilds目录
\item 对于amd64和x86的平台的用户需要选择 current-install-amd64-minimal/ 或 current-install-x86-minimal/ 目录。如果需要所有其它平台的,请进入 current-iso/ 目录。
\end{enumerate}

下载之后要对下载好的iso文件进行校验防止到时候出现无法使用的问题。


\subsubsection{下载stage3文件}
进入镜像站的/gentoo/releases/amd64/autobuilds/目录
如果你对systemd没有则进入current-stage3-amd64/目录选择最新的stage3下载到本地。


\subsection{在Linux下制作安装介质}

\begin{lstlisting}[language=Bash]
  dd if=gentoo.iso of=/dev/sdb bs=10M
\end{lstlisting}
\subsection{在Windows上准备安装介质}
可以使用软碟通之类的软件进行刻录

\chapter{配置网络}

这里默认使用的dhcpcd服务分配的网络。

测试是否可以联通外网
\begin{lstlisting}
  ping -c4 google.com
\end{lstlisting}

\chapter{磁盘分区}
\section{分区表}
\begin{tabular}{|l|r|r|r|}
  \hline
  物理位置 & 挂载点 & 文件系统 & 大小 \\
  \hline
  /dev/sda1 & /boot & ext2 & 200M \\
  \hline
  /dev/sda2 & swap & swap & 8G \\
  \hline
  /dev/sda3 & / & ext4 & 20G \\
  \hline
\end{tabular}

swap 分区可以按照你的物理内存大小*2来进行划分,如果你的内存大于32G 也可以选择不分swap,但是还是建议单独分出来swap之后的休眠会用到。 
\section{分区}

\begin{lstlisting}[language=Bash]
  fdisk /dev/sda
\end{lstlisting}

\subsection{格式化分区}

\begin{lstlisting}[language=Bash]
  mkfs.ext2 /dev/sda1
  mkswap /dev/sda2
  mkfs.ext4 /dev/sda3 
\end{lstlisting}

\subsection{挂载分区}
\begin{lstlisting}[language=Bash]
  mkdir -pv /mnt/gentoo
  mount /dev/sda2 /mnt/gentoo
  mkdir -pv /mnt/gentoo/boot
  mount /dev/sda1 /mnt/gentoo/boot
\end{lstlisting}

\chapter{安装stage3文件}

将下载好的stage3 文件放到 /mnt/gentoo/目录下解压缩

\begin{lstlisting}
  tar -xvpf stage*
\end{lstlisting}

\chapter{安装基本系统}

\begin{lstlisting}
  mkdir --parents /mnt/gentoo/etc/portage/repos.conf
  cp /mnt/gentoo/usr/share/portage/config/repos.conf /mnt/gentoo/etc/portage/repos.conf/gentoo.conf
\end{lstlisting}

\section{进入chroot}
\begin{lstlisting}
  cp --dereference /etc/resolv.conf /mnt/gentoo/etc/

  mount --types proc /proc /mnt/gentoo/proc
  mount --rbind /sys /mnt/gentoo/sys
  mount --make-rslave /mnt/gentoo/sys
  mount --rbind /dev /mnt/gentoo/dev
  mount --make-rslave /mnt/gentoo/dev
  test -L /dev/shm && rm /dev/shm && mkdir /dev/shm
  mount --types tmpfs --options nosuid,nodev,noexec shm /dev/shm
  chmod 1777 /dev/shm
  chroot /mnt/gentoo /bin/bash
  source /etc/profile
  export PS1="(chroot) ${PS1}"


  
\end{lstlisting}

\subsection{同步portage树}
可以使用
\begin{lstlisting}
  emerge-webrsync
\end{lstlisting}
来进行快速同步。

也可以在同步完成之后进行增量同步

\begin{lstlisting}
  emerge --sync
\end{lstlisting}

\subsection{阅读新闻}
\begin{lstlisting}
  eselect news list
  eselect news read
\end{lstlisting}

\subsection{选择系统profile}
\begin{lstlisting}
  eselect profile list
  eselect profile set
    [16]  default/linux/amd64/17.1 (stable) *
\end{lstlisting}
\subsection{可选 Portage Tmpfs}
为了加快编译的速度可以将配置文件放在内存中编译。
\begin{lstlisting}
  tmpfs /var/tmp         tmpfs rw,nosuid,noatime,nodev,size=20G,mode=1777 0 0
tmpfs /var/tmp/portage tmpfs rw,nosuid,noatime,nodev,size=40G,mode=775,uid=portage,gid=portage,x-mount.mkdir=775 0 0
\end{lstlisting}


\subsection{升级系统}
\begin{lstlisting}
  emerge --ask --verbose --update --deep --newuse @world
\end{lstlisting}

等到这些运行完了之后别着急再运行下面这几条
\begin{lstlisting}
  emerge @preserved-rebuild
  perl-cleaner --all
  emerge -auvDN --with-bdeps=y @world
\end{lstlisting}

\chapter{配置系统}


\section{时间和地区}
\begin{lstlisting}
echo "Asia/Shanghai" > /etc/timezone
emerge --config sys-libs/timezone-data
echo "en_US.UTF-8 UTF-8
zh_CN.UTF-8 UTF-8" >> /etc/locale.gen
locale-gen
eselect locale list

\end{lstlisting}

安装必要的包

\begin{lstlisting}
  emerge -av gentoo-sources genkernel 
\end{lstlisting}
\section{配置主机名称}
\begin{lstlisting}
  echo hostname=\"Test\" > /etc/conf.d/hostname
\end{lstlisting}

\chapter{配置系统工具}
\begin{lstlisting}
  emerge -av app-admin/sysklogd sys-process/cronie sudo layman grub
  sed -i 's/\# \%wheel ALL=(ALL) ALL/\%wheel ALL=(ALL) ALL/g' /etc/sudoers
  passwd
  rc-update add sysklogd default
  rc-update add cronie default
\end{lstlisting}

\chapter{增加管理员用户}
\begin{lstlisting}
  useradd -m -G users,wheel,portage,usb,video chris
\end{lstlisting}

\chapter{配置内核}
配置文件可以看 当前的文件夹的 etc/kernel/.config

\begin{lstlisting}
  make oldconfig
  make -j40
  make modules_install
  make install
  
\end{lstlisting}
\chapter{网络管理}
\begin{lstlisting}
  emerge -av networkmanager
  rc-update add NetworkManager default
\end{lstlisting}

\section{配置fstab}


\begin{lstlisting}
  wget https://raw.githubusercontent.com/YangMame/Gentoo-Installer/master/genfstab
  chmod +x genfstab
  mv genfstab /usr/bin
  genfstab -U / > /etc/fstab 
\end{lstlisting}

\section{安装grub}

\begin{lstlisting}
  emerge -av grub
  grub-install /dev/sda
  grub-mkconfig -o /boot/grub/grub.cfg
\end{lstlisting}


\chapter{清理系统}
\begin{lstlisting}
  rm -f /stage3-*
\end{lstlisting}


\end{document}

