
\chapter{磁盘分区}
\section{分区表}
\begin{tabular}{|l|r|r|r|}
  \hline
  物理位置 & 挂载点 & 文件系统 & 大小 \\
  \hline
  /dev/sda1 & /boot & ext2 & 200M \\
  \hline
  /dev/sda2 & swap & swap & 8G \\
  \hline
  /dev/sda3 & / & ext4 & 20G \\
  \hline
\end{tabular}

swap 分区可以按照你的物理内存大小*2来进行划分,如果你的内存大于32G 也可以选择不分swap,但是还是建议单独分出来swap之后的休眠会用到。 
\section{分区}

\begin{lstlisting}[language=Bash]
  fdisk /dev/sda
\end{lstlisting}

\subsection{格式化分区}

\begin{lstlisting}[language=Bash]
  mkfs.ext2 /dev/sda1
  mkswap /dev/sda2
  mkfs.ext4 /dev/sda3 
\end{lstlisting}

\subsection{挂载分区}
\begin{lstlisting}[language=Bash]
  mkdir -pv /mnt/gentoo
  mount /dev/sda2 /mnt/gentoo
  mkdir -pv /mnt/gentoo/boot
  mount /dev/sda1 /mnt/gentoo/boot
\end{lstlisting}
