

\chapter{安装基本系统}

\begin{lstlisting}
  mkdir --parents /mnt/gentoo/etc/portage/repos.conf
  cp /mnt/gentoo/usr/share/portage/config/repos.conf /mnt/gentoo/etc/portage/repos.conf/gentoo.conf
\end{lstlisting}

\section{进入chroot}
\begin{lstlisting}
  cp --dereference /etc/resolv.conf /mnt/gentoo/etc/

  mount --types proc /proc /mnt/gentoo/proc
  mount --rbind /sys /mnt/gentoo/sys
  mount --make-rslave /mnt/gentoo/sys
  mount --rbind /dev /mnt/gentoo/dev
  mount --make-rslave /mnt/gentoo/dev
  test -L /dev/shm && rm /dev/shm && mkdir /dev/shm
  mount --types tmpfs --options nosuid,nodev,noexec shm /dev/shm
  chmod 1777 /dev/shm
  chroot /mnt/gentoo /bin/bash
  source /etc/profile
  export PS1="(chroot) ${PS1}"


  
\end{lstlisting}

\subsection{同步portage树}
可以使用
\begin{lstlisting}
  emerge-webrsync
\end{lstlisting}
来进行快速同步。

也可以在同步完成之后进行增量同步

\begin{lstlisting}
  emerge --sync
\end{lstlisting}

\subsection{阅读新闻}
\begin{lstlisting}
  eselect news list
  eselect news read
\end{lstlisting}

\subsection{选择系统profile}
\begin{lstlisting}
  eselect profile list
  eselect profile set
    [16]  default/linux/amd64/17.1 (stable) *
\end{lstlisting}
\subsection{可选 Portage Tmpfs}
为了加快编译的速度可以将配置文件放在内存中编译。
\begin{lstlisting}
  tmpfs /var/tmp         tmpfs rw,nosuid,noatime,nodev,size=20G,mode=1777 0 0
tmpfs /var/tmp/portage tmpfs rw,nosuid,noatime,nodev,size=40G,mode=775,uid=portage,gid=portage,x-mount.mkdir=775 0 0
\end{lstlisting}


\subsection{升级系统}
\begin{lstlisting}
  emerge --ask --verbose --update --deep --newuse @world
\end{lstlisting}

等到这些运行完了之后别着急再运行下面这几条
\begin{lstlisting}
  emerge @preserved-rebuild
  perl-cleaner --all
  emerge -auvDN --with-bdeps=y @world
\end{lstlisting}
